%% ne pas utiliser l'indentation automatique sinon ça va décaler le code sur la
%% droite et ça ne tiendra pas sur le pdf
Équivalences de code Java, Scala :\\
\noindent
\small\addtolength{\tabcolsep}{-5pt}
\begin{tabular}{|c|c|}
  \hline
  \textbf{Java} & \textbf{Scala}\\
  \hline
  \lstset{language=Java}
  \begin{lstlisting}
boolean hasUpperCase = false
for (int i = 0; i < name.length(); i++)
  if (Character.isUpperCase(name.charAt(i))) {
    hasUpperCase = true;
    break;
  }
\end{lstlisting} &   \lstset{language=Scala}
\begin{lstlisting}
// Utilisation des collections et closures Scala
val hasUpperCase = name.exists(_.isUpper)
\end{lstlisting}
\\
\hline
\lstset{language=Java}
\begin{lstlisting}
public class Person {
  private String name;
  private int age;
  public Person(String name, int age) {
    this.name = name; this.age = age;
  }

  public String getName() { return name;}
  public int getAge() { return age; }

  public void setName(String name) { 
    this.name = name; 
  }
  public void setAge(int age) { 
     this.age = age; 
  }
}
\end{lstlisting} &
\lstset{language=Scala}
\begin{lstlisting}
// Scala créé automatiquement les getters et
// setters pour 'name' et 'age'
class Person(var name: String, var age: Int)
\end{lstlisting}
\\
\hline
\end{tabular}
