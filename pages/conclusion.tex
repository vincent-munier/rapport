\chapter{Conclusion}

Ce stage fut très enrichissant au niveau technique.\\
J'ai appris à utiliser de nombreux outils évolués (SBT, AWS, Mercurial)
et j'ai également amélioré mes compétences par la pratique pour ceux que je
connaissais déjà (Scala).

Lors de mon stage précédent chez Normation, je développais déjà en Scala et je
souhaitais continuer à l'utiliser. C'est ce langage Scala qui m'a attiré et
c'est pour pouvoir accroître mes connaissances sur cette technologie que j'ai
voulu travailler chez Mimesis Republic.\\

Après mes deux premiers mois de stage chez Mimesis Republic, l'application web
est rendue disponible.
Les premiers retours sont positifs. De nombreuses idées d'amélioration font
aussi surface de la part des nouveaux utilisateurs.
Ces améliorations ont été implémentées au cours du stage.

L'outil web est maintenant utilisé quotidiennement par les différentes équipes
de la société. C'est aujourd'hui l'outil le plus utilisé pour gérer son
infrastructure. En dehors de l'outil web, j'ai aussi contribué activement à
l'évolution de l'outil d'infra en ligne de commande.
Avant mon départ, j'ai également formé mon chef d'équipe qui sera mon successeur
pour maintenir l'outil.

Les objectifs du stage ont été atteints et même dépassés car aujourd'hui non
seulement l'appli Web est disponible pour toutes les équipes, mais j'ai
également complètement pris en charge la maintenance de l'outil. 
La raison d'être d'un stage de fin d'études est d'assurer la transition de
l'université à l'entreprise. Pour ma part, cette transition s'est effectuée avec
succès.

À l'avenir, je souhaite travailler pour une entreprise semblable à Mimesis
Republic. C'est-à-dire travailler sur un projet ambitieux, aux côtés de
programmeurs expérimentés et dans une équipe de taille moyenne tout en restant
dans le cadre dynamique qui est celui des start-up.



