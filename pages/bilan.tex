
\chapter{Bilan du stage}
En prenant du recul sur mon travail, je tire les enseignements suivants :
\begin{itemize}
  \item[\textbullet] déployer une version stable de l'application le plus
    régulièrement possible permet d'avoir plus de retours de la part des
    utilisateurs.

  \item[\textbullet] proposer aux clients de l'application de donner leur avis et leur
    suggestions. Il ne doit y avoir aucune retenue pour faire une proposition
    de fonctionnalité. Même si une fonctionnalité ne sera pas implémentée car le
    coût temps/intérêt est trop fort ou qu'il y a d'autres tâches plus
    prioritaires, il est toujours intéressant d'en discuter.

    Exemple de proposition non implémentée : avoir un système d'authentification
    connecté à Active Directory pour accéder à la webapp. Il aurait ainsi été
    possible de savoir qui exécute une commande et autoriser l'accès depuis
    internet via authentification. La webapp est aujourd'hui accessible
    uniquement en intranet.
    Le trafic entrant est filtré pour se connecter à la webapp. Seuls les postes
    de travail de l'entreprise faisant partie d'une liste spécifique d'IP y ont accès.
    Ce système d'authentification n'a pas été implémenté car d'autres tâches
    étaient plus prioritaires.

  \item[\textbullet] avant d'entreprendre l'implémentation d'une idée personnelle, il est
    préférable de demander aux équipiers ce qu'ils en pensent. Tout le monde
    n'est pas forcément d'accord sur l'intérêt qu'elle peut avoir.
\end{itemize}
C'est donc ma communication que je devrais principalement améliorer.

Faire son stage dans une start-up requiert d'avoir une bonne autonomie.
Les employés ont déjà beaucoup de travail et problèmes qu'ils doivent régler,
aider un stagiaire n'est pas la plus haute priorité. Avec les nombreux projets
effectués durant mon parcours, l'université m'a appris à être autonome et
trouver des solutions. Si tout de même un problème persistait, je pouvais
demander de l'aide à la personne la plus informée qui n'hésitait pas apporter
son aide.

À propos de façon de travailler, c'est la première fois que j'ai fait du
\textit{pair programming} en entreprise.
Cette méthode de travail consiste à programmer à deux devant le même écran,
l'un écrit le code tandis que l'autre passe en revue chaque ligne de code et
fait des suggestions. Cette technique de développement est très utile lorsqu'il
s'agit de transmettre la connaissance d'une base de code à l'autre.

Ce stage m'a aussi donné l'occasion de travailler avec des informaticiens
passionnés et talentueux qui n'hésitent pas à partager leurs connaissances.
C'est un environnement dynamique dans lequel il faut beaucoup travailler mais
qui permet de progresser rapidement pour devenir meilleur développeur.

Tout n'est pas parfait pour autant chez Mimesis Republic.
Le jeu n'est pas encore très répandu. Il est plus stimulant de travailler
sur un projet populaire, connu de tous. Programmer un mini-jeu qui sera
 utilisé par des centaines de milliers de personnes est valorisant pour le
 développeur.

À certains moments, on peut sentir la pression des dirigeants lorsque les
objectifs en terme de nombre de visiteurs n'ont pas été atteints.
Ceci peut venir d'un ordre conceptuel, le jeu ne plaît pas assez.
Ou bien d'un ordre technique, il n'est plus possible de rentrer dans la 3D par
exemple.
Lorsqu'un problème technique grave est rencontré, il doit être réparé dans
rapidement car une heure d'inaccessibilité de la 3D peut faire perdre de
nombreux joueurs qui ne reviendront plus.
Ce sont donc de lourdes responsabilités qui sont confiées aux différentes
équipes qui sont parfois contraintes de rester tard pour corriger un problème
urgent. Ceci est juste une observation, en tant que stagiaire, de tels
responsabilités ne m'ont pas été attribuées.
Il y a beaucoup de travail et beaucoup d'attentes.

