
\chapter{Le contexte : l'outil de déploiement du jeu en ligne}

Les ingénieurs de l'équipe \textit{infrastructure} ont conçu un outil en ligne
de commande pour déployer une infra.
Chaque équipe de l'entreprise utilise cet outil pour gérer leur propre
infrastructure. Par exemple, l'équipe \textit{engine} utilise l'outil de
déploiement pour gérer leur infrastructure \textit{engine}.

Un développeur de l'équipe \textit{engine} peut exécuter la commande
\textit{status engine} pour se renseigner sur l'état de son infra. L'outil en
ligne de commande affiche :
\begin{verbatim}
engine_all_0 (i-51e60a18 / 54.247.21.95)	:	running
Instance reachability check passed 
System reachability check passed
\end{verbatim}
Tout va bien, l'instance \verb?engine_all_0? est bien en train de s'exécuter.

\section{L'outil d'infra et AWS}

L'outil d'infra utilise l'API Java AWS pour :
\begin{itemize}
  \item créer, supprimer une instance
  \item récupérer le status d'une instance
  \item démarrer, arrêter une instance
  \item créer, supprimer un groupe de sécurité
  \item authoriser des IP et ports sur un groupe de sécurité
  \item lister les groupes de sécurité
\end{itemize}

\section{L'outil d'infra et Chef}

Toutes les informations concernants une infra sont stockées dans Chef.
Lors de la création d'une infra, l'outil en ligne de commande stocke entre
autres le nom de l'infra, la région AWS, le nom de la base de données et le nom
du groupe de sécurité dans un Databag Chef.

L'outil d'infra utilise l'API REST de Chef pour récupèrer des informations ou
effectuer des actions. 

\section{Usage}

L'outil d'infra en ligne de commande reçoit en entrés un nom de commande
et ses arguments. Si la commande n'existe pas, l'outil d'infra affiche l'aide
d'usage.
Voici l'affichage de l'aide, on peut y voir toutes les commandes disponibles :
\begin{verbatim}
Usage: infra <cmd> [<infra>] [<file>] [<key>=<value>]*

   Infra admin tool config file:
     Config parameters must be specified in the 'infra.conf' file located in the 'config' directory.
     It should contain one <field>=<value> definition per line.

   Infra creation, pass config file:
     create <file>
       Create an infrastructure as defined in the given config file.
    migrate <file>
       Migrate databases for a particular environment in an infrastructure as
       defined in the given config file.


   Infra management, pass infra:
     start <infra>
       Start the given infrastructure.
     stop <infra>
       Stop the given infrastructure.
     status <infra>
       Show status of the given infrastructure.
     destroy <infra>
       Destroy the given infrastructure.

   Service management, pass infra:
     start-services <infra>
       Start services on the given infrastructure.
     stop-services <infra>
       Stop services on the given infrastructure.
     restart-services <infra>
       Restart services on the given infrastructure.
     status-services <infra>
       Show status of the services on the given infrastructure.

   Database management, pass infra:
     deploy-db <infra>
       Run the 'deployDb' command on the given infrastructure.
     liquibase-install <infra>
       Run the 'liquibase-install' command on the given infrastructure.
     liquibase-sync <infra>
       Run the 'liquibase-sync' command on the given infrastructure.
     liquibase-update <infra>
       Run the 'liquibase-update' command on the given infrastructure.
     run-migration-tool <infra>
       Run the 'run-migration-tool' command on the given infrastructure.

   Text management, pass infra :
     deploy-wti <infra>
       Export the last texts from WTI on amazon S3 ; 
       to really use them you have to modify the file "launch.js.php" on the "a" environment.

   Environment management, pass infra and optionally updated values:
     show <infra> [<key>]*
       Show the "a" environment on the given infrastructure.
     update-env <infra> <file>
       Update the "a" environment on the given infrastructure as defined in the given config file.
     update-env <infra> [<key=value>]*
       Update the "a" environment on the given infrastructure as defined in the key-value arguments.

   Various utilities, rarely used:
     list
       List all infrastructures.
     start-all
       Start all infrastructures and all services.
     stop-all
       Stop all infrastructures.
     register-dns <infra>
       Register DNS records for the given infrastructure.
     switch-dns <infra>
       Switch LIVE dns to point to the given infrastructure.
     export <infra>
       Export the databag item describing the given infrastructure from the Chef server to a file.
     import <infra>
       Import the databag item describing the given infrastructure from a file to the Chef server.
     export-databags
       Export all databags from the Chef Server to the file system.
     import-databags
       Import all databags from the file system to the Chef Server.
\end{verbatim}
