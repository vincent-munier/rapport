\usepackage{textcomp}
\usepackage{listings}
\usepackage{color}

\definecolor{lightgray}{rgb}{.9, .9, .9}
\definecolor{darkgray}{rgb}{.4, .4, .4}
\definecolor{purple}{rgb}{0.65, 0.12, 0.82}

\lstdefinelanguage{JavaScript}{
  keywords={typeof, new, true, false, catch, function, return, null, catch, switch, var, if, in, while, do, else, case, break},
  keywordstyle=\color{blue}\bfseries,
  ndkeywords={class, export, boolean, throw, implements, import, this},
  ndkeywordstyle=\color{darkgray}\bfseries,
  identifierstyle=\color{black},
  sensitive=false,
  comment=[l]{//},
  morecomment=[s]{/*}{*/},
  commentstyle=\color{purple}\ttfamily,
  stringstyle=\color{red}\ttfamily,
  morestring=[b]',
  morestring=[b]",
  extendedchars=true,
  basicstyle=\footnotesize\ttfamily,
  showstringspaces=false,
  showspaces=false,
  frame=single,                   % adds a frame around the code
  %% numbers=left,
  numberstyle=\footnotesize,
  numbersep=9pt,
  tabsize=2,
  breaklines=true,
  showtabs=false,
  captionpos=b,
  %% lstlisting ne gère pas les accents (c'est écrit dans la doc).
  %% Pour contourner ce problème, on peut énumérer les caractères qu'on va utiliser
  %% avec literate et extendedchars=true.
  %% Pour chaque accent il faut placer le caractère entre accolades (ex {à}) et
  %% ensuite écrire la traduction entre double accolades (ex {{à}}) et enfin écrire
  %% le nombre (1) . Entre deux entrées, on peut mettre des espaces pour plus de
  %% clareté.
  extendedchars=true,
  literate={é}{{\'e}}1 {è}{{\`e}}1 {à}{{\`a}}1 {ç}{{\c{c}}}1 {œ}{{\oe}}1 {ù}{{\`u}}1
  {É}{{\'E}}1 {È}{{\`E}}1 {À}{{\`A}}1 {Ç}{{\c{C}}}1 {Œ}{{\OE}}1 {Ê}{{\^E}}1
  {ê}{{\^e}}1 {î}{{\^i}}1 {ô}{{\^o}}1 {û}{{\^u}}1 {dollar}{{\$}}1
}


