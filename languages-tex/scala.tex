\definecolor{javared}{rgb}{0.6,0,0} % for strings
\definecolor{javagreen}{rgb}{0.25,0.5,0.35} % comments
\definecolor{javapurple}{rgb}{0.5,0,0.35} % keywords
\definecolor{javadocblue}{rgb}{0.25,0.35,0.75} % javadoc

\lstdefinelanguage{Scala}{
  morekeywords={%
          abstract,case,catch,class,def,do,else,extends,%
          false,final,finally,for,forSome,if,implicit,import,lazy,%
          match,new,null,object,override,package,private,protected,%
          return,sealed,super,this,throw,trait,true,try,type,%
          val,var,while,with,yield},
  otherkeywords={=>,<-,<\%,<:,>:,\#,@},
  sensitive=true,
  keywordstyle=\color[rgb]{0,0,1},
  morecomment=[l]{//},
  morecomment=[n]{/*}{*/},
  basicstyle=\ttfamily,
  keywordstyle=\color{javapurple}\bfseries,
  stringstyle=\color{javared},
  commentstyle=\color{javagreen},
  morestring=[b]",
  morestring=[b]',
  morestring=[b]""",
  %% lstlisting ne gère pas les accents (c'est écrit dans la doc).
  %% Pour contourner ce problème, on peut énumérer les caractères qu'on va utiliser
  %% avec literate et extendedchars=true.
  %% Pour chaque accent il faut placer le caractère entre accolades (ex {à}) et
  %% ensuite écrire la traduction entre double accolades (ex {{à}}) et enfin écrire
  %% le nombre (1) . Entre deux entrées, on peut mettre des espaces pour plus de
  %% clareté.
  literate={é}{{\'e}}1 {è}{{\`e}}1 {à}{{\'a}}1 {ç}{{\c{c}}}1 {œ}{{\oe}}1 {ù}{{\`u}}1
  {É}{{\'E}}1 {È}{{\`E}}1 {À}{{\`A}}1 {Ç}{{\c{C}}}1 {Œ}{{\OE}}1 {Ê}{{\^E}}1
  {â}{{\^a}}1 {ê}{{\^e}}1 {î}{{\^i}}1 {ô}{{\^o}}1 {û}{{\^u}}1 {dollarSign}{{\$}}1
  {percentageSign}{{\%}}1
}[keywords,comments,strings]

